% !Mode:: "TeX:UTF-8"

\titlecontents{chapter}[2em]{\vspace{.5\baselineskip}\xiaosan\song}%
             {\prechaptername\CJKnumber{\thecontentslabel}\postchaptername\qquad}{} %
             {}             % 设置该选项为空是为了不让目录中显示页码          
\addcontentsline{toc}{chapter}{外文资料}
%\setcounter{page}{1}       % 如果需要从该页开始从 1 开始编页,则取消该注释
\markboth{外文资料}{外文资料}
\chapter*{外文资料}
Virtualization is a technological enabler used on a large
spectrum of applications, ranging from cloud computing and
servers to mobiles and embedded systems [1]. As a fundamental cornerstone of cloud computing, virtualization provides
numerous advantages for workload management, data protection, and cost-/power-effectiveness [2]. On the other side of the
spectrum, the embedded and safety-critical systems industry
has resorted to virtualization as a fundamental approach to
address the market pressure to minimize size, weight, power,
and cost (SWaP-C), while guaranteeing temporal and spatial
isolation for certification (e.g., ISO26262) [3]–[5]. Due to the
proliferation of virtualization across multiple industries and
use cases, prominent players in the silicon industry started
to introduce hardware virtualization support in mainstream
computing architectures (e.g., Intel Virtualization Technology,
Arm Virtualization Extensions, respectively) [6], [7].


Recent advances in computing architectures have brought to
light a novel instruction set architecture (ISA) named RISCV [8]. RISC-V has recently reached the mark of 10+ billion
shipped cores [9]. It distinguishes itself from the classical
mainstream, by providing a free and open standard ISA,
featuring a modular and highly customizable extension scheme
that allows it to scale from small microcontrollers up to
supercomputers [10]–[16]. The RISC-V privileged architecture
provides hardware support for virtualization by defining the
Hypervisor extension [17], ratified in Q4 2021.


Despite the Hypervisor extension ratification, as of this
writing, there is no RISC-V silicon with this extension on
the market1
. There are open-source hypervisors with upstream
support for the Hypervisor extension, i.e., Bao [1], Xvisor [18],
KVM [19], and seL4 [20] (and work in progress in Xen [21]
and Jailhouse [22]). However, to the best of our knowledge,
there are just a few hardware implementations deployed on
FPGA, which include the Rocket chip [23] and NOEL-V [24]
(and soon SHAKTI and Chromite [25]). Notwithstanding, no
existing work has (i) focused on understanding and enhancing
the microarchitecture for virtualization and (ii) performed a
design space exploration (DSE) and accompanying power,
performance, area (PPA) analysis.


In this work, we describe the architectural and microarchitectural support for virtualization in a open source RISC-V
CVA6-based [14] (64-bit) SoC. At the architectural level, the
implementation is compliant with the Hypervisor extension
(v1.0) [17] and includes the implementation of the RISC-V
timer (Sstc) extension [26] as well. At the microarchitectural
level, we modified the vanilla CVA6 microarchitecture to
support the Hypervisor extension, and proposed a set of
additional extensions / enhancements to reduce the hardware
virtualization overhead: (i) a L1 TLB, (ii) a dedicated second
stage TLB coupled to the PTW (i.e., GTLB in our lingo), and
(iii) an L2 TLB. We also present and discuss a comprehensive
design space exploration on the microarchitecture. We first
evaluate 23 (out of 288) hardware designs deployed on FPGA
(Genesys 2) and assess the impact on functional performance
(execution cycles) and hardware. Then, we elect 10 designs
and analyze them in depth with post-layout simulations of
implementations in 22nm FDX technology.

To summarize, with this work, we make the following
contributions. Firstly, we provide hardware virtualization support in the CVA6 core. In particular, we design a set of
(virtualization-oriented) microarchitectural enhancements to
the nested memory management unit (MMU) (Section III). To
the best of our knowledge, there is no work or study describing
and discussing microarchitectural extensions to improve the
hardware virtualization support in a RISC-V core. Second,
we perform a design space exploration (DSE), encompassing
dozens of design configurations. This DSE includes tradeoffs on parameters from three different microarchitectural
components (L1 TLB, GTLB, L2 TLB) and respective impact
on functional performance and hardware costs (Section IV).
Finally, we conduct post-layout simulations on a few elected
design configurations to assess a Power, Performance, and
Area (PPA) analysis (Sections V).

For the DSE evaluation, we ran the MiBench (automotive subset) benchmarks to assess functional performance.
The virtualization-aware non-optimized CVA6 implementation
served as our baseline configuration. The software setup encompassed a single Linux VM running atop Bao hypervisor for
a single-core design. We measured the performance speedup
of the hosted Linux VM relative to the baseline configuration.
Results from the DSE exploration demonstrated that the proposed microarchitectural extensions can achieve a functional
performance speedup up to 19% (e.g., for the susanc small
benchmark); however, in some cases at the cost of a nonnegligible increase in area and power with respect to the
baseline. Thus, results from the PPA analysis show that: (i)
the Sstc extension has negligible impact on power and area;
(iii) the GTLB increases the overall area in less than 1%;
and (iv) the L2 TLB introduces a non-negligible 8% increase
in area in some configurations. As a representative, wellbalanced configuration, we selected the CVA6 design with Sstc
support and a GTLB with 8 entries. For this specific hardware
configuration, we observed a performance speedup of up to
16% (approx. 12.5% on average) at the cost of 0.78% in area
and 0.33% in power. To the best of our knowledge, this paper
reports the first public work on a complete DSE evaluation
and PPA analysis for a virtualization-enhanced RISC-V core.

\section*{Background}

\subsection*{RISC-V Privileged Specification}

The RISC-V privileged instruction set architecture (ISA)
[17] divides its execution model into 3 privilege levels: (i)
machine mode (M-mode) is the most privileged level, hosting
the firmware which implements the supervisor binary interface (SBI) (e.g., OpenSBI); (ii) supervisor mode (S-Mode)
runs Unix type operating systems (OSes) that require virtual
memory management; and (iii) user mode (U-Mode) executes
userland applications. The modularity offered by the RISC-V
ISA seamlessly allows for implementations at distinct design
points, ranging from small embedded platforms with just Mmode support, to fully blown server class systems with M/S/U.

\subsection*{CVA6}

CVA6 (formerly known as Ariane) is an application class
RISC-V core that implements both the RV64 and RV32
versions of RISC-V ISA [14]. The core fully supports the 3 privilege execution modes M/S/U-modes and provides hardware support for memory virtualization by implementing a
memory management unit (MMU), making it suitable for
running a fully-fledged OS such as Linux. Recent additions
also include a energy-efficient Vector Unit co-processor [27].
Internally, the CVA6 microarchitecture encompasses a 6-stage
pipeline, single issue, with an out-of-order execution stage and
8 PMP entries. The MMU has separate TLBs for data and
instructions and the Page Table Walker (PTW) implements the
Sv39 and Sv32 translation modes as defined by the privileged
specification [17].

\subsection*{RISC-V Virtualization}

Unlike other mainstream ISAs, the RISC-V privileged architecture was designed from the initial conception to be
classically virtualizable [28]. So although, the ISA, per se,
allows the straightforward implementation of hypervisors resorting for example to classic virtualization techniques []
(e.g., trap-and-emulation and shadow page tables), it is well
understood that such techniques incur in a prohibitive performance penalty and cannot cope with current embedded
real-time virtualization requirements (e.g, interrupt latency)
[23]. Thus, to increase virtualization efficiency, the RISCV privileged architecture specification introduced hardware
support for virtualization through the (optional) Hypervisor
extension [17].

\textbf{Privilege Levels.} As depicted in Figure 1, the RISC-V Hypervisor extension execution model follows an orthogonal
design where the supervisor mode (S-mode) is modified to an
hypervisor-extended supervisor mode (HS-mode) well-suited
to host both type-1 or type-2 hypervisors2
. Additionally, two
new privileged modes are added and can be leveraged to run
the guest OS at virtual supervisor mode (VS-mode) and virtual
user mode (VU-mode).


\textbf{Two-stage Address Translation} The Hypervisor extension
also defines a second stage of translation (G-stage in RISC-V
lingo) to virtualize the guest memory by translating guestphysical addresses (GPA) into host-physical addresses (HPA).
The HS-mode operates like S-mode but with additional hypervisor registers and instructions to control the VM execution and G-stage translation. For instance, the hgatp register
holds the G-stage root table pointer and respective translation
specific configuration fields.

\textbf{Hypervisor Control and Status Registers (CSRs).} Each VM
running in VS-mode has its own control and status registers
(CSRs) that are shadow copies of the S-mode CSRs. These
registers can be used to determine the guest execution state
and perform VM switches. To control the virtualization state,
a specific flag called virtualization mode (V bit) is used. When
V=1, the guest is executing in VS-mode or VU-mode, normal
S-mode CSRs accesses are actually accessing the VS-mode
CSRs, and the G-stage translation is active. Otherwise, if
V=0, normal S-mode CSRs are active, and the G-stage is
disabled. To ease guest-related exception trap handling, there
are guest specific traps, e.g., guest page faults, VS-level illegal
exceptions, and VS-level ecalls (a.k.a. hypercalls).

\subsection*{Nested-MMU}

The MMU is a hardware component responsible for translating virtual memory references to physical ones, while enforcing memory access permissions. The OS holds control over the
MMU by assigning a virtual address space to each process and
managing the MMU translation structures in order to correctly
translate virtual addresses (VAs) into physical addresses (PAs).
On a virtualized system, the MMU can translate from guestvirtual addresses (GVAs) to guest-physical addresses (GPAs)
and from GPA into host-physical addresses (HPAs). In this
case, this feature is referred to as nested-MMU. The RISCV ISA supports the nested-MMU through a new stage of
translation that converts GPA into HPA, denoted G-stage. The
guest VM takes control over the first stage of translation
(VS-stage in RISC-V lingo), while the hypervisor assumes
control over the second one (G-stage). Originally, the RISCV privileged specification defines that a VA is converted into
a PA by transversing a multi-level radix-tree table using one
of four different topologies: (i) Sv32 for 32 virtual address
spaces (VAS) with a 2-level hierarchy tree; (ii) Sv39 for 39-
bit VAS with a 3-level tree; (iii) Sv48 for 48-bit VAS and
4-level tree; and (iv) Sv57 for 57-bit VAS and 5-level tree.
Each level holds a pointer to the next table (non-leaf entry) or
the final translation (leaf entry). This pointer and respective
permissions are stored in a 64-bit (RV64) or 32-bit (RV32)
width page table entry (PTE). Note that RISC-V splits the
virtual address into 4KiB page sizes, but since each level can
either be a leaf on non-leaf, it supports superpages to reduce
the TLB pressure, e.g., Sv39 supports 4KiB, 2MiB, and 1GiB
page sizes.

\subsection*{RISC-V "stimecmp/vstimecmp" Extension(Sstc)}

The RISC-V Sstc extension [26] aims at enhancing supervisor mode with its own timer interrupt facility, thus eliminating
the large overheads for emulating S/HS-mode timers and
timer interrupt generation up in M-mode. The Sstc extension
also adds a similar facility to the Hypervisor extension for
VS-mode. To enable direct control over timer interrupts in the HS/S-mode and VS-mode, the Sstc encompasses two
additionally CSRs: (i) stimecmp and (ii) vstimecmp. Whenever
the value of time counter register is greater then the value
of stimecmp, the supervisor timer interrupt pending (STIP)
bit goes high and a timer interrupt is deliver to HS/S-mode.
The same happens for VS-mode, with one subtle difference:
the offset of the guest delta register (htimedelta) is added to
the time value. If vstimecmp is greater than time+htimedelta,
VSTIP goes high and the VS-mode timer interrupt is generated.
For a complete overview of the RISC-V timer architecture and
a discussion on why the classic RISC-V timer specification incurs a significant performance penalty, we refer the interested
reader to [23].

\section*{CVA6 Hypervisor Support: Architecture And Microarchitecture}

In this section, we describe the architectural and microarchitectural hardware virtualization support in the CVA6 (compliant with the RISC-V Hypervisor extension v1.0), illustrated
in Figure 2.

\subsection*{A. Hypervisor and Virtual Supervisor Execution Modes}

As previously described (refer to Section II-C), the Hypervisor extension specification extends the S-mode into the HSmode and adds two extra orthogonal execution modes, denoted
VS-mode and VU-mode. To add support for this new execution
modes, we have extended/modified some of the CVA6 core
functional blocks, in particular, the CSR and Decode modules.
As illustrated by Figure 2, the hardware virtualization architecture logic encompasses five building blocks: (i) VS-mode
and HS-mode CSRs access logic and permission checks; (ii)
exceptions and interrupts triggering and delegation; (iii) trap
entry and exit; (iv) hypervisor instructions decoding and execution; and (v) nested-MMU translation logic. The CSR module was extended to implement the first three building blocks
that comprise the hardware virtualization logic, specifically:
(i) HS-mode and VS-mode CSRs access logic (read/write
operations); (ii) HS and VS execution mode trap entry and
return logic; and (iii) a fraction of the exception/interrupt
triggering and delegation logic from M-mode to HS-mode
and/or to VS/VU-mode (e.g., reading/writing to vsatp CSR
triggers an exception in VS-mode when VTM bit is set on
hstatus CSR). The Decode module was modified to implement
hypervisor instructions decoding (e.g., hypervisor load/store
instructions and memory-management fence instructions) and
all VS-mode related instructions execution access exception
triggering.

We refer readers to Table I, which presents a summary
of the features that were fully and partially implemented.
We have implemented all mandatory features of the ratified
1.0 version of the specification; however, we still left some
optional features as partially implemented, due to the dependency on upcoming or newer extensions. For example:
hvencfg bits depend on Zicbom [29] (cache block management
operations); hgeie and hgeip depend on the Advanced Interrupt
Architecture (AIA) [30]; and hgatp depends on virtual address
spaces not currently supported in the vanilla CVA6.

\subsection*{B. Hypervisor Load/Store Instructions}

The hypervisor load/store instructions (i.e., HLV, HSV, and
HLVX) provide a mechanism for the hypervisor to access
the guest memory while subject to the same translation and
permission checks as in VS-mode or VU-mode. These instructions change the translation settings at the instruction
granularity level, forcing a full swap of privilege level and
translation applied at every hypervisor load/store instruction
execution. The implementation encompasses the addition of
a signal (identified as hyp ld/st in Figure 2) to the CVA6
pipeline that travels from the decoding to the load/store unit
in the execute stage. This signal is then fed into the MMU
that performs (i) all necessary context switches (i.e., enables
the hgatp and vstap CSRs), (ii) enables the virtualization
mode, and (iii) changes execution mode as specified in the
hstatus.SPVP field.

\subsection*{C. Nested Page-Table Walker(PTW)}

One of the major functional blocks of an MMU is the pagetable walker (PTW). Fundamentally, the PTW is responsible
for partitioning a virtual address accordingly to the specific
topology and scheme (e.g., Sv39) and then translating it into a
physical address using the memory page tables structures. The
Hypervisor extension specifies a new stage of translation (Gstage) that is used to translate guest-physical addresses into
host-physical addresses. Our implementation supports Bare
translation mode (no G-stage) and Sv39x4, which defines a
41-bit width maximum guest physical address space (virtual
space already supported by the CVA6).

We extended the existing finite state machine (FSM) used to
translate VA to PA and added only a new control state to keep
track of the current stage of translation and assist the context
switching between VS-Stage and G-Stage translations. With
the G-stage in situ, it is mandatory to translate (i) the leaf GPA
resulting from the VS-Stage translation but also (ii) all nonleaf PTE GPA used during the VS-Stage translation walk. To
accomplish that, we identify three stages of translation that can
occur during a PTW iteration: (i) VS-Stage - the PTW current
state is translating a guest virtual address into a GPA; (ii) GStage Intermed - the PTW current state is translating non-leaf
PTE GPA into HPA; and (iii) G-Stage Final - the PTW current
state is translating the final output address from VS-Stage into
an HPA. It is worth noting if hgatp is in Bare mode, no Gstage translation is in place, and we perform a standard VSStage translation. Once the nested walk completes, the PTW
updates the TLB with the final PTE from VS-stage and Gstage, alongside the current address space identifier (ASID)
and the VMID. One implemented optimization consists in
storing the translation page size (i.e., 4KiB, 2MiB, and 1GiB)
for both VS- and G-stages into the same TLB entry as well
as permissions access bits for each stage. This helps to reduce
the TLB hit time and improve hardware reuse.

\subsection*{D. Virtualization-aware TLbs(vTLB)}

The CVA6 MMU microarchitecture has two small fully
associative TLB: one for data (L1 DTLB) and the other for
instructions (L1 ITLB). Both TLBs support a maximum of 16
entries and fully implement the flush instructions, i.e., sfence,
including filtering by ASID and virtual address. To support
nested translation, we modified the microarchitecture of the L1
DTLB and ITLB to support two stages of translation, including
access permissions and VMIDs. Each TLB entry holds both
VS-Stage and G-Stage PTE and respective permissions. The
lookup logic is performed using the merged final translation
size from both stages, i.e. if the VS-stage is a 4KiB and
the G-stage is a 2MiB translation, the final translation would
be a 4KiB. This is probably one of the major drawbacks of
having both the VS-stage and G-stage stored together. For
instance, hypervisors supporting superpages/hugepages use the
2MiB page size to optimize the MMU performance. Although
this significantly reduces the PTW walk time, if the guest
uses 4KiB page size, the TLB lookup would not benefit
from superpages, since the translation would be stored as a
4KiB page size translation. The alternative approach would
be to have separate TLBs for the VS-stage and G-stage final
translations, but it would translate into higher hardware costs,
less hardware reuse (if G-stage is not active), and a higher
performance penalty on an L1 TLB hit (TLB search time
increased by approximately a factor of 2). Since L1 TLBs are
fully combinational circuits that lie on the critical path of the
CPU, we decide to keep the VS-stage and G-stage translations
in a single TLB entry. Finally, the TLB also supports VMID
tags allowing hypervisors to perform a more efficient TLB
management using per-VMID flushes, and avoiding full TLB
flush on a VM context switch. As a final note, the TLB also
allows flushes by guest physical address, i.e., hypervisor fence
instructions (HFENCE.VVMA/GVMA) are fully supported.

\subsection*{E. Microarchitectural extension \#1 - GTLB}

A full nested table walk for the Sv39 scheme can take up
to 15 memory accesses, five times more than a standard Sstage translation. This additional burden imposes a higher TLB
miss penalty, resulting in a (measured) overhead of up to 13%
on the functional performance (execution cycles) (comparing
the baseline virtualization implementation with the vanilla
CVA6). To mitigate this, we have extended the CVA6 microarchitecture with a G-Stage TLB (GTLB) in the nested-PTW
module to store intermediate GPA to HPA translations, i.e.,
VS-Stage non-leaf PTE guest physical address to host physical
addresses translation. Figure 3 illustrates the modifications
required to integrate the GTLB in the MMU microarchitecture.
The GTLB structure aims at accelerating VS-stage translation
by skipping each nested translation during the VS-stage page
table walk. Figure 4 presents a 4KiB page size translation
process featuring a GTLB in the PTW. Without the GTLB,
each time the guest forms the non-leaf physical address PTE
pointer during the walk, it needs: (i) to translate it via Gstage, (ii) read the next level PTE value from memory, and
(iii) resume the VS-stage translation. When using superpages
(2MiB or 1GiB), there are fewer translation walks, reducing
the performance penalty. With a simple hardware structure, it
is possible to mitigate such overheads by keeping those Gstage translations cached, while avoiding unnecessary cache
pollution and nondeterministic memory accesses. Another
reason to support such an approach is related to the fact that
PTEs are packed together in sequence in memory, i.e., multiple
VS-Stage non-leaf PTE address translations will share the
same GTLB entry [31].




