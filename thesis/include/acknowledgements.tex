% !Mode:: "TeX:UTF-8"
\titleformat{\chapter}{\centering\xiaosan\hei}{\chaptername}{2em}{} % 标题四号黑体居中

% \titlecontents{chapter}[2em]{\vspace{.5\baselineskip}\xiaosan\song}
%              {\prechaptername\CJKnumber{\thecontentslabel}\postchaptername\qquad}{} 
%              {}                            % 设置该选项为空是为了不让目录中显示页码


% \fancypagestyle{plain}{
%     \fancyhf{}
%     \fancyhead[C]{\song\wuhao 天津大学 2017 届本科生毕业论文}
% }           

\markboth{致\qquad 谢}{致\qquad 谢}
\addcontentsline{toc}{other}{致\quad 谢} % 添加到目录中
\chapter*{致\quad 谢}

当我写到这里的时候,预示着我本科最后一门课程即将完成,大学四年本科经历即将圆满结束。在大学四年的期间,我从一名对计算机一窍不通的学生变成了可以独立完成复杂任务的计算机研究者。本项目从立项到开题再到代码编写以及最后的论文撰写总共历时五个月的时间,在这段时间内,我遇到了一些困难,但同时也从外界获得了许多帮助,使我能够综合运用本科阶段所学知识,并最终坚持完成毕业设计。在完成毕业设计的过程中,我也学到了很多的知识,并获得了成长。

首先要感谢我的毕业设计指导老师李罡老师,李老师在我毕设的进行过程中一直非常地关注我的进展,我们多次在微信上进行交流。除此之外,李老师也对我的论文写作进行了检查与指导,使我能够更好地完成论文的撰写。在李老师的指导下,我明白了如何严谨地撰写一篇科研文章。同时,李老师也对学生的想法有很大的包容与支持。

除此之外,我还要感谢清华大学的陈渝老师,陈老师在我毕设进行过程中十分关注我的进展与成果,并且在陈老师的推荐下,我得以参加第十届开源操作系统年度技术会议发表了有关我的毕业设计的演讲,
并且和国内顶级的系统专家进行了交流。陈老师也阅读了我的论文并且给出了十分中肯的建议,这对于我完善毕业设计与论文有着十分重要的作用。

接着,我还要感谢在我毕设过程中帮助过我的人,包括清华大学的贾越凯师兄与郑鈜壬师兄、重庆大学的陈泱宇师兄和华中科技大学的蒋周奇师兄,他们在我毕设遇到困难的时候给予了我很多帮助,使得我得以克服这些困难并且坚持完成了复杂系统的设计。

除此之外,我也要感谢葡萄牙米尼奥大学的 José Martins 博士以及 Rivos 公司的 Dylan Reid 工程师,他们分别是 bao-hypervisor 以及 salus 的作者,我从开源社区知道了他们,当我开发遇到困难时曾用邮件向他们请求帮助,他们十分热情地回复了我并凭借他们多年的经验与卓越的工程能力帮助我解决了问题,在这里对他们致以诚挚的感谢!

同时,我还要感谢魏继增魏老师在大学期间对我的指导与照顾。魏老师十分关注对于学生的培养且十分尊重学生的意见。我作为天津大学为数不多的做系统与体系结构相关研究的学生,魏老师不仅在学业上指导帮助我,还在我对于前途迷茫的时候为我答疑解惑。十分感谢魏老师大学四年以来对我的帮助!


最后,请允许我向我的父母致以最诚挚的谢意。正是他们的支持与鼓励,使得我能够在大学四年中专心学习,尽情去追求自己喜欢的事物。
同时,感谢父母对我的理解与安慰,在疫情期间我经常感到焦虑,情绪不稳定,正是有他们的安慰与鼓励,包容我的缺点与不足,我才能取得今天的成果。

最后的最后,请允许我回忆我的大学本科四年生涯,在大学四年,我从一个懵懵懂懂的高中生,不知道自己的未来在哪里,最终蜕变成了一个可以独立思考、有未来清晰目标的本科毕业生。 在大一时,我加入了学校的互联网工作室并学习了基础的编程知识,培养了我对于计算机的兴趣;在大二时,我参加了全国大学生操作系统大赛,在比赛中,我学会了如何写一个系统, 在比赛的过程中我遇到了很多志同道合的朋友并且坚定了之后的职业发展方向。在大三与大四过程中,我继续深入学习课程知识, 参与了研究生入学考试并且最终被中国科学院大学录取,在之后将继续做系统与体系结构方向的研究。 在写这篇致谢的时候,我正在清华大学操作系统实验室实习,未来将继续朝着目标前进。


\clearpage